\documentclass[12pt, oneside]{article}
\usepackage[a6paper,margin=5mm]{geometry}
\usepackage{blindtext}
\title{Manual testing interview questions}
\begin{document}
\author{Vishnu}
\maketitle
\section{Bug life cycle}
\begin{itemize}

\item New: When a new defect is logged and posted for the first time. 
    It is assigned as NEW.
\item Assigned: Once the bug is posted by the tester, 
    the lead of the tester approves the bug and assigns the 
    bug to the development team
\item Duplicate: If the defect is repeated twice or the defect 
    corresponds to the same concept of the bug, 
    the status is changed to "duplicate."
\item Rejected: If the developer feels the defect is not a genuine defect then it changes the defect to "rejected."
\item Deferred: If the present bug is not of a prime priority and if it is expected to get fixed in the next release, then status "Deferred" is assigned to such bugs
\item Not a bug:If it does not affect the functionality of the application then the status assigned to a 
\item Open: The developer starts analyzing and works on the defect fix
\item Fixed: When a developer makes a necessary code change and verifies the change, he or she can make bug status as "Fixed."
\item Pending retest: Once the defect is fixed the developer gives a particular code for retesting the code to the tester. Since the software testing remains pending from the testers end, the status assigned is "pending request."
\item Retest: Tester does the retesting of the code at this stage to check whether the defect is fixed by the developer or not and changes the status to "Re-test."
\item Verified: The tester re-tests the bug after it got fixed by the developer. If there is no bug detected in the software, then the bug is fixed and the status assigned is "verified."
\item Reopen: If the bug persists even after the developer has fixed the bug, the tester changes the status to "reopened". Once again the bug goes through the life cycle.
\item Closed: If the bug is no longer exists then tester assigns the status "Closed." 
\end{itemize}
\section{Bug reporting}

\section{Bug severity}
Severity is defined as the degree of impact a Defect has on the development or operation of a component application being tested.
Types:Critical,Major, Medium, Low

\section{Bug priority}

Priority is defined as the order in which a defect should be fixed. Higher the priority the sooner the defect should be resolved.
Types:High, Medium, Low

\section{High priority low severity}
Ex:Report Name is wrong 

\section{Low priority high severity}
Ex:Date picker allows future dates to be entered in all reports
\section{Can we write test case with out use case document} 


\section{Black box testing methodology}
Black Box Testing is a software testing method in which the internal structure/ design/ implementation of the item being tested is NOT known to the tester

The following are various blackbox testing methodologies

\begin{itemize}
    \item Boundary value Analysis
    \subitem Boundary testing is the process of testing between extreme ends or boundaries between partitions of the input values.
    \subitem So these extreme ends like Start- End, Lower- Upper, Maximum-Minimum, Just Inside-Just Outside values are called boundary values and the testing is called "boundary testing".
    \item Equivalent Partitioning 

\end{itemize}
\section{White box testing methodology}

\section{Difference between retesting and regression testing}
\subsection{Re-Testing:}
After a defect is detected and fixed, the software should be retested to confirm that the original defect has been successfully removed. This is called Confirmation Testing or Re-Testing
\subsection{Regression testing:}
Testing your software application when it undergoes a code change to ensure that the new code has not affected other parts of the software.

\section{Data driven testing:}
Different data set is feeded for same script for different scenarios
Ex:Book Appointment -->script
Book appointment for different departments -->scenario
we will feed separate data set for each department for same script



\end{document}
